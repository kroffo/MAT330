\documentclass{scrartcl}
\usepackage{amsmath,amssymb}
\setkomafont{disposition}{\normalfont\bfseries}

\title{Abstract Algebra}
\subtitle{Homework 4: 1, 5.3, 5.4, 5.14, 5.22}
\author{Kenny Roffo}
\date{Due February 20}

\begin{document}

\maketitle

1. Let $S$ be the group of all $3\times 3$ permutation matrices. Determine all of the subgroups of $S$. Determine which subgroups are cyclic and find a generator. Those which are not cyclic, prove whether they are abelian. (Naming of sets for reference in subgroup lattice)

\begin{displaymath}
S=\left \{
\begin{bmatrix}
1 & 0 & 0 \\
0 & 1 & 0 \\
0 & 0 & 1 
\end{bmatrix},
\begin{bmatrix}
1 & 0 & 0 \\
0 & 0 & 1 \\
0 & 1 & 0 
\end{bmatrix},
\begin{bmatrix}
0 & 1 & 0 \\
1 & 0 & 0 \\
0 & 0 & 1 
\end{bmatrix},
\begin{bmatrix}
0 & 1 & 0 \\
0 & 0 & 1 \\
1 & 0 & 0 
\end{bmatrix},
\begin{bmatrix}
0 & 0 & 1 \\
1 & 0 & 0 \\
0 & 1 & 0 
\end{bmatrix},
\begin{bmatrix}
0 & 0 & 1 \\
0 & 1 & 0 \\
1 & 0 & 0 
\end{bmatrix}
\right \}
\end{displaymath}

This subgroup of $S$ is neither cyclic nor abelian. Consider the counterexample \begin{displaymath}
\begin{bmatrix}
0 & 0 & 1 \\
1 & 0 & 0 \\
0 & 1 & 0 
\end{bmatrix}\begin{bmatrix}
1 & 0 & 0 \\
0 & 0 & 1 \\
0 & 1 & 0
\end{bmatrix}=\begin{bmatrix}
0 & 1 & 0 \\
1 & 0 & 0 \\
0 & 0 & 1 
\end{bmatrix}\end{displaymath} but commuting the matrices we see \begin{displaymath}\begin{bmatrix}
1 & 0 & 0 \\
0 & 0 & 1 \\
0 & 1 & 0
\end{bmatrix}\begin{bmatrix}
0 & 0 & 1 \\
1 & 0 & 0 \\
0 & 1 & 0
\end{bmatrix}=\begin{bmatrix}
0 & 0 & 1 \\
0 & 1 & 0 \\
1 & 0 & 0
\end{bmatrix}
\end{displaymath} Thus this subgroup of $S$ is nonabelian.\\

\begin{displaymath}
S_1=\left \{
\begin{bmatrix}
1 & 0 & 0 \\
0 & 1 & 0 \\
0 & 0 & 1 
\end{bmatrix}
\right \}
\end{displaymath}

This subgroup of $S$ is cyclic and abelian. It contains one element, the identity, so this is trivial.\\

\begin{displaymath}
S_2=\left \{
\begin{bmatrix}
1 & 0 & 0 \\
0 & 1 & 0 \\
0 & 0 & 1 
\end{bmatrix},\begin{bmatrix}
1 & 0 & 0 \\
0 & 0 & 1 \\
0 & 1 & 0 
\end{bmatrix}
\right \}
\end{displaymath}
This subgroup of $S$ is cyclic with generator $\begin{bmatrix}
1 & 0 & 0 \\
0 & 0 & 1 \\
0 & 1 & 0 
\end{bmatrix}$.\\

\begin{displaymath}
S_3=\left \{
\begin{bmatrix}
1 & 0 & 0 \\
0 & 1 & 0 \\
0 & 0 & 1 
\end{bmatrix},\begin{bmatrix}
0 & 0 & 1 \\
0 & 1 & 0 \\
1 & 0 & 0 
\end{bmatrix}
\right \}
\end{displaymath}
This subgroup is cyclic with generator $\begin{bmatrix}
0 & 0 & 1 \\
0 & 1 & 0 \\
1 & 0 & 0 
\end{bmatrix}$.\\

\begin{displaymath}
S_4=\left \{
\begin{bmatrix}
1 & 0 & 0 \\ 0 & 1 & 0 \\ 0 & 0 & 1
\end{bmatrix},\begin{bmatrix}
0 & 1 & 0 \\ 1 & 0 & 0 \\ 0 & 0 & 1
\end{bmatrix}
\right \}
\end{displaymath}
This subgroup is cyclic with generator $\begin{bmatrix}
0 & 1 & 0 \\
1 & 0 & 0 \\
0 & 0 & 1 
\end{bmatrix}$.\\
\begin{displaymath}
S_5=\left \{
\begin{bmatrix}
1 & 0 & 0 \\
0 & 1 & 0 \\
0 & 0 & 1
\end{bmatrix},\begin{bmatrix}
0 & 1 & 0 \\
0 & 0 & 1 \\
1 & 0 & 0
\end{bmatrix},\begin{bmatrix}
0 & 0 & 1 \\
1 & 0 & 0 \\
0 & 1 & 0
\end{bmatrix}
\right \}
\end{displaymath}
This subgroup of $S$ is cyclic with generator $\begin{bmatrix}
0 & 0 & 1 \\
1 & 0 & 0 \\
0 & 1 & 0
\end{bmatrix}$.\\

5.3: $SL(n,\mathbb{R})$ is the set of all $n\times n$ matrices with determinant equal to 1. Prove that $SL(n,\mathbb{R})$ is a subgroup of $GL(n,\mathbb{R})$.\\

\emph{Proof}: $GL(n,\mathbb{R})$ is the set of all invertible $n\times n$ matrices. By the invertible matrix theorem, $GL(n,\mathbb{R})$ is the set of all $n\times n$ matrices with nonzero determinants. By definition, every element of $SL(n,\mathbb{R})$ has determinant equal to 1, which is nonzero. Therefore every element of $SL(n,\mathbb{R})$ is also an element of $GL(n,\mathbb{R})$. Thus to show $SL(n,\mathbb{R})$ is a subgroup of $B$ we need only show it is closed under matrix multiplication and inverses. Let $A,B \in SL(n,\mathbb{R})$. From the properties of matrices, we know for an invertible matrix $X$ that $det(X^{-1})=\frac{1}{det(X)}$. $det(A)=1$ so $det(A^{-1})=\frac{1}{1}=1$. Therefore $A^{-1} \in SL(n,\mathbb{R})$. Also, from the properties of matrices we know for matrices $X,Y$ that $det(XY)=det(X)det(Y)$. Then $det(AB)=det(A)det(B)=(1)(1)=1$, so $AB \in SL(n,\mathbb{R})$. Thus we have shown $SL(n,\mathbb{R})$ is closed under inverses and matrix multiplication, therefore, by definition of subgroup, $SL(n,\mathbb{R})$ is a subgroup of $GL(n,\mathbb{R})$.\pagebreak.

5.4: Find all the subgroups of each group and sketch the corresponding subgroup lattice:\\\\
a) $(\mathbb{Z}_8,+)$:\\\\
$<1>=\{1,2,3,4,5,6,7,0\}$\\
$<2>=\{2,4,6,0\}$\\
$<4>=\{4,0\}$\\
$<0>=\{0\}$\\\\
b) $(\mathbb{Z}_{35},+)$:\\\\
$<1>=\{1,2,3,...,33,34,0\}$\\
$<5>=\{5,10,15,20,25,30,0\}$\\
$<7>=\{7,14,21,28,0\}$\\
$<0>=\{0\}$\\\\
c) $(\mathbb{Z}_{36},+)$:\\\\
$<1>=\{1,2,3,...,34,35,0\}$\\
$<2>=\{2,4,6,...,32,34,0\}$\\
$<3>=\{3,6,9,...,30,33,0\}$\\
$<4>=\{4,8,16,...,28,32,0\}$\\
$<6>=\{6,12,18,24,30,0\}$\\
$<9>=\{9,18,27,0\}$\\
$<12>=\{12,24,0\}$\\
$<18>=\{18,0\}$\\
$<0>=\{0\}$\pagebreak

5.14: Prove that the intersection of two subgroups of a group $G$ is itself a subgroup of $G$.\\

\emph{Proof}: Let $H$ and $K$ be subgroups of some group $G$. Then all elements of $H$ and $K$ are subsets of $G$ which are closed under the operation of $G$ and inverses. Since $H$ and $K$ are subsets of $G$, $H \cap K$ is also a subset of $G$. We must show $H \cap K$ is closed under the operation of $G$. Let $X,Y \in H \cap K$. Then $X,Y \in H$, and $X,Y \in K$. Since $H$ and $K$ are subgrouops of $G$, they are closed under the operation, so $XY \in H$ and $XY \in K$. Then $XY \in H \cap K$.  We must show $H \cap K$ is closed under inverses. Let $X \in H\cap K$. Then $X \in H$ and $X \in K$. Since $H$ and $K$ are subgroups of $G$, they are closed under inverses, so $X^{-1} \in H$, and $X^{-1} \in K$. Since $X^{-1}$ is in both $H$ and $K$, $X^{-1} \in H \cap K$, thus $H \cap K$ is closed under inverses. We have shown $H \cap K$ is closed under the operation of $G$ and inverses, thus we have shown $H \cap K$ is a subgroup of $G$.\\\\

5.22: Let $G$ be a group. Prove that $Z(G)$ is a subgroup of $G$.\\

\emph{Proof}: $Z(G)$ is the subset of elements $z$ in $G$ for which $zx=xz$ for all $x \in G$. Since $G$ is a group, the operation, $*$, in on $G$ is an associative binary operator. Also, the identity, $e$ is an element of $Z(G)$ since the identity of a group is commutative by definition of a group. Thus to prove $Z(G)$ is a group we must show $Z(G)$ is closed under * and inverses. Let $w,z \in Z(G)$ and let $x$ be an arbitrary element of $G$. Then $zx=xz$ and $wx=xw$ by definition of $Z(G)$. Multiplying on the left and right sides by $z^{-1}$, an element of $G$, we have $z^{-1}zxz^{-1}=z^{-1}xzz^{-1}$, which simplifies to $xz^{-1}=z^{-1}x$. Thus $z^{-1}$ is commutative in $G$, so $z^{-1}$ satisfies the condition to be an element of $Z(G)$. Therefore inverses exist in $Z(G)$ for all elements in $Z(G)$. Also, $zwx=zxw$ (by multiplying on the left by $z$. Since $G$ is a group and thus closed under its operation, $xw \in G$, so $zwx=xwz$ since $z \in Z(G)$. Now since $w \in G$, we see $wz=zw$, so $zwx=xzw$. Therefore $zw$ is commutative with $x$, and arbitrary element in $G$. So $Z(G)$ is closed under the operation of $G$. Thus all of the group axioms are satisfied by $Z(G)$ so $Z(G)$ is a group. Therfore, since all elements of $Z(G)$ are elements of $G$, $Z(G)$ is a subgroup of $G$.
 



\end{document}
