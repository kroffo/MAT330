\documentclass{scrartcl}
\usepackage{amsmath,amssymb}
\setkomafont{disposition}{\normalfont\bfseries}

\title{Abstract Algebra}
\subtitle{Homework 3: 3.15, 3.16, 1, 4.20, 4.21}
\author{Kenny Roffo}
\date{Due February 13 (Late)}

\begin{document}

\maketitle

3.15: Let $G$ be a nonempty set and let $*$ be an associative binary operation on $G$. Assume that both the left and right cancellation laws hold in $(G,*)$. Assume moreover that $G$ is finite. Show that $(G,*)$ is a group.\\

\em Proof: \em To show $(G,*)$ is a group, we must show that $(G,*)$ has an identity, and inverses. We will first show the existence of the identity element:

Since $G$ is finite, $G$ has some number of elements, $n$, each denoted $x_i$. Consider an element, $x_1$, and the corresponding list of elements of $G$, ${x_1, x_1 * x_1, x_1 * x_2, ..., x_1 * x_n}$, which has $n+1$ elements. Since every member of this list is an element of $G$, and there are more elements in the list than elements in $G$, it follows that there must exist some $x_j$ such that $x_1*x_j=x_1$. Now assume $\exists x_t$ such that $x_j * x_t \ne x_t$. This is true if and only if $x_1 * x_j * x_t \ne x_1 * x_t \iff x_1 * x_t \ne x_1 * x_t$, which is obviously false. Therefore it must be the case that $\forall x_i \in G$, $x_j * x_i = x_i$. Likewise, assume $\exists x_t$ such that $x_t * x_j \ne x_t$. This is true if and only if $x_t * x_j * x_1 \ne x_t * x_1 \iff x_t * x_1 = x_t * x_1$, which is false. Thus it must be the case that $\forall x_i \in G$, $x_i * x_j = x_i$. Therefore, there exists an element, $x_j$, such that $x_i * x_j = x_j * x_i = x_i$, $\forall x_i \in G$. Thus $(G,*)$ has an identity element.

We must now show that $G$ has inverses. Let $x_i$ be an arbitrary element of $G$. Then the list $x_i, x_i * x_i, ... , x_i^n$ contains $n+1$ elements. Since $G$ has only $n$ elements, two elements of the list must be equal, $x_i^s=x_i^t=x_i^t*e$, where s>t. Using the cancellation laws, this expression simplifies to $x_i*x_i^{s-t-1}=x_i^{s-t}=x_i=e$, which implies by definition that $x_i^{s-t-1}=x_i^{-1}$. Therefore, for every $x_i \in G$, $\exists x_i^{_1}$.

Therefore, all of the group axioms are satisfied by $(G,*)$, so $(G,*)$ is a group.\\

3.16: Consider the nonegative integers under multiplication. multiplication is an associative binary operator on the set, and the cancellation laws hold, but though 1 is the identity, inverses do not exist for all elements. Consider $2$, which has inverse $1/2$. $1/2$ is not a nonnegative integer, so the inverse for $2$ is not in the set. Therefore the nonegative integers under multiplication do not form a group.\pagebreak

1: Find a group, $G$, with $(x*y)^{-1} \ne x^{-1}*y^{-1} $ $ \forall x,y \in G$:\\

Consider the group $GL(2,\mathbb{R})$ under matrix multiplication. Let $x=\begin{pmatrix} 1 & 1 \\ 1 & 2 \end{pmatrix}$ and $y=\begin{pmatrix} 1 & 2 \\ 1 & 1 \end{pmatrix}$. Then $x^{-1}=\begin{pmatrix} -1 & 1 \\ 2 & -1 \end{pmatrix}$ and $y^{-1}=\begin{pmatrix} -1 & 2 \\ 1 & -1\end{pmatrix}$. We see $x*y=\begin{pmatrix} 1 & 1 \\ 1 & 2 \end{pmatrix}\cdot\begin{pmatrix} 1 & 2 \\ 1 & 1 \end{pmatrix}=\begin{pmatrix} 2 & 3 \\ 3 & 5 \end{pmatrix}$. Taking the inverse we have $(x*y)^{-1}=\begin{pmatrix} 5 & -3 \\ -3 & 2 \end{pmatrix}$. Also $x^{-1}*y^{-1}=\begin{pmatrix} -1 & 1 \\ 2 & -1 \end{pmatrix}\cdot\begin{pmatrix} -1 & 2 \\ 1 & -1\end{pmatrix}=\begin{pmatrix} 2 & -3 \\ -3 & 5 \end{pmatrix}$. Notice that $(x*y)^{-1} \ne x^{-1} * y^{-1}$, so we have found a group such that it is not the case for all elements $x,y$ in the group that $(x*y)^{-1}=x^{-1} * y^{-1}$. \\

4.20: Let $G$ be a group and let $a \in G$. An element $b \in G$ is called a \em conjugate\em of $a$ if there exists an element $x \in G$ such that $b=xax^{-1}$. Show that any conjugate of $a$ has the same order as $a$.\\

Let $a,b,y \in G$ such that $a=wbw^{-1}$. Then $b$ is a conjugate of $a$. Let $r=o(a)$. We must show $o(wbw^{-1})=r$. First we will show that for some $x,y$ in any group, $(yxy^{-1})^m=yx^my^{-1}=e$ for all $m>0$ by mathematical induction.

Let $P(n)$ be the statement $(yxy^{-1})^m=yx^my^{-1}$. $P(1)$ is true since $(yxy^{-1})^1=yxy^{-1}$ and $yx^1y^{-1}=yxy^{-1}$, and $yxy^{-1}=yxy^{-1}$. Let $k\ge 1$, and assume $P(k)$ is true. That is, assume $(yxy^{-1})^k=yx^ky^{-1}$. We see $(yxy^{-1})^{k+1}=(yxy^{-1})^k(yxy^{-1})=(yx^ky^{-1})(yxy^{-1})=yx^ky^{-1}yxy^{-1}=yx^kxy^{-1}=yx^{k+1}y^{-1}$. That is, $(yxy^{-1})^{k+1}=yx^{k+1}y^{-1}$, thus $P(k)$ implies $P(k+1)$. Therefore $P(n)$ is true for all $n\ge 1$. Thus for all integers $m>0$, $(yxy^{-1})^m=yx^my^{-1}=e$.

Now that the above equality has been proven, we have that $(wbw^{-1})^r=wb^rw^{-1}=wew^{-1}=e$. Thus, $o(wbw^{-1})\le r$. Now let $z=wbw^{-1}$. Then $b=w^{-1}zw$, and letting $t=w^{-1}$, $b=tzt^{-1}$, thus $b$ is a conjugate of $z$. Therefore, $o(b)\le o(z) = o(wbw^{-1}) = o(a) = r \le o(b)$. Thus $o(wbw^{-1}=r)$, so a conjugate of $a$ has the same order as $a$.
\\ 

4.21: Show that for any two elements $x, y$ of any group $G$, $o(xy)=o(yx)$:\\

\em Proof: \em For this proof we will consider two cases:\\

Case 1: Assume $o(xy)=\infty$. Assume for the sake of contradiction $o(yx)\ne \inf$. Then $\exists n \in \mathbb{Z}$ such that $o(yx)=n$. Then $(yx)(yx)(yx)..$(n times)$..(yx) = e$. Multiplying both sides by $x$ on the left, we have $x(yx)(yx)(yx)..$(n times)$..(yx) = x*e = x$, and by associativity $(xy)(xy)(xy)..$(n times)$..(xy)x = x$. Applying the right cancellation law, this implies $(xy)(xy)(xy)..$(n times)$..(xy) = e$. But this implies $o(xy)=n$, which contradicts our assumption that $o(xy)=\infty$. Therefore if $o(xy)=\infty$ then it must be the case that 

Case 2: Assume $o(xy)=n$ for some $n \in \mathbb{Z}$. Then $(xy)(xy)(xy)..$(n times)$..(xy) = e$. Multiplying by y on the left to both sides syields $y(xy)(xy)(xy)..$(n times)$..(xy) = y$, and by associativity, $(yx)(yx)(yx)..$(n times)$..(yx)y = y$. By the right cancellation law, we have $(yx)(yx)(yx)..$(n times)$..(yx) = e$ Thus it follows that $(yx)^n=e$. If we can show no integer $k<n$ exists such that $(yx)^k=e$, then we will have $o(yx)=n$. For the sake of contradiction, assume such a $k$ exists. Then $(yx)(yx)(yx)..$(k times)$..(yx) = e$. Multiplying both sides by $x$ on the left, we have $x(yx)(yx)(yx)..$(k times)$..(yx) = x*e = x$, and by associativity $(xy)(xy)(xy)..$(k times)$..(xy)x = x$. Applying the right cancellation law, this implies $(xy)(xy)(xy)..$(k times)$..(xy) = e$. But this implies $o(xy)=k$, which contradicts our assumption that $o(xy)=n$. Therefore no such $k<n$ exists, so $o(yx)=n$.

We have shown that given $x,y \in G$, $o(xy)=o(yx)$.

\end{document}
