\documentclass{scrartcl}
\usepackage{amsmath,amssymb}
\setkomafont{disposition}{\normalfont\bfseries}

\title{Abstract Algebra}
\subtitle{Homework 7: 11.4, 11.7, 11.15, 11.16, 11.17, 11.18, 11.29}
\author{Kenny Roffo}
\date{Due April 13}

\newcommand{\normal}{\unlhd}

\begin{document}
\maketitle

\textbf{11.4)} Let $H\normal G$ and $K\normal G$. Show that $H\cap K\normal G$.\\

$H \normal G$ and $K \normal G$ implies $\forall g \in G, h \in H, k \in K$, $ghg^{-1} \in H$ and $gkg^{-1} \in K$. $\forall a \in H \cap K$, we know $a \in H$, so $\forall g \in G, gag^{-1} \in H$. Similarly, $\forall g \in G, gag^{-1} \in K$. Thus $gag^{-1} \in H \cap K, \forall g \in G$. Now to show $H \cap K \normal G$ we must only show that $H \cap G$ is a group. We do so by showing $H \cap K$ has inverses, associativity, and the identity. Since $H$ and $K$ are both subgroups of $G$, they contain the identity of $G$, so the identity $e$ is in $H \cap K$. Now let $x,y,z \in H \cap K$. Then $x,y,z \in H$, and $(xy)z=x(yz)$ since $H$ is a group, so $H \cup K$ has associativity. Lastly, we see that $x \in H, x \in K$ implies $x^{-1} \in H, x^{-1} \in K$, since $H$ and $K$ are groups, so $x^{-1} \in H \cap K$. Thus $H \cap K$ is a group, and displays the properties required to be considered normal, so $H \cap K \normal G$\\

\textbf{11.7)} Let $H \normal G$ and $K \normal G$ and assume that $H \cap K = \{e\}$. Show that if $x \in H$ and $y \in K$ then $xy=yx$.\\

Let $x \in H, y \in K$. Then $\forall g \in G, gxg^{-1} \in H, gyg^{-1} \in K$. Since $H$ and $K$ are subgroups of $G$, this implies $yxy^{-1} \in H, xyx^{-1} \in K$, and therefore $x(yxy^{-1}) \in H$ and $(xyx^{-1})y \in K$. Thus $xyx^{-1}y^{-1} \in H \cap K$, which implies
\begin{align*}
xyx^{-1}y^{-1}&=e\\
xyx^{-1}&=y\\
xy&=yx
\end{align*}
Therefore for any two normal subgroups of a group $G$ with intersection ${e}$ all elements of one subgroup commute with all elements of the other.\pagebreak

\textbf{11.15} $D_4/Z(D_4)$ is ``just like'' one of the groups with which you are familiar. Which one?\\

Let's start by finding the cardinality of this group, which is equal to the number of right cosets of $Z(D_4)$ in $D_4$. $D_4=\{e,r,r^2,r^3,s,rs,r^2s,r^3s\}$, and $Z(D_4)=\{e,r^2\}$. By LaGrange's Theorem we have $\frac{|G|}{|H|}=[G:H]=\frac{8}{2}=4$. We know there are only two groups of order 4, so this is either ``just like'' the Klein-4 group or $\mathbb{Z}_4$, which is cyclic. Let's list the right cosets of $Z(D_4)$ in $D_4$:
\begin{align*}
Z(D_4)&=\{e,r^2\}\\
Z(D_4)r&=\{r,r^3\}\\
Z(D_4)r^2&=\{e,r^2\}\\
Z(D_4)r^3&=\{r,r^3\}\\
Z(D_4)s&=\{s,r^2s\}\\
Z(D_4)rs&=\{rs,r^3s\}\\
Z(D_4)r^2s&=\{s,r^2s\}\\
Z(D_4)r^3s&=\{rs,r^3s\}\\
\end{align*}
Thus the right cosets of $Z(D_4)$ in $D_4$ are $Z(D_4)=Z(D_4)r^2, Z(D_4)r=Z(D_4)r^3, Z(D_4)s=Z(D_4)r^2s$ and $Z(D_4)rs=Z(D_4)r^3s$. It is now apparent that $D_4/Z(D_4)$ is not cyclic (neither of the cosets without $s$ can generate the ones with $s$, and both the ones with $s$ are of order 2). Thus $D_4/Z(D_4)$ is ``just like'' the Klein-4 group.\\

\textbf{11.16)} Show that $(\mathbb{Q},+)/(\mathbb{Z},+)$ is an infinite group every element of which has finite order.\\

Every element of $(\mathbb{Q},+)/(\mathbb{Z},+)$ can be written as $\mathbb{Z}+\frac{p}{q}$ where $p$ and $q$ are integers. By choosing $p=1$ and $q \in \mathbb{Z}$ we already have an infinite number of elements, though there are many more elements than these. Now, let $\mathbb{Z}+\frac{p}{q}$ be an arbitrary element of the group. We see $(\mathbb{Z}+\frac{p}{q})^2=\mathbb{Z}+2\frac{p}{q}$, and likewise $(\mathbb{Z}+\frac{p}{q})^3=\mathbb{Z}+3\frac{p}{q}$. More generally, $(\mathbb{Z}+\frac{p}{q})^n=\mathbb{Z}+n\frac{p}{q}$. Notice if we choose $n=q$ then we have $(\mathbb{Z}+\frac{p}{q})^q=\mathbb{Z}+q\frac{p}{q}=\mathbb{Z}+p=\mathbb{Z}$. Thus the order of $\mathbb{Z}+\frac{p}{q}$ is the finite integer $q$, and since every element can be written in this way, every element is of finite order.\pagebreak

\textbf{11.17)} Let $G$ be abelian and let $H$ be a subgroup of $G$. Show that $G/H$ is abelian.\\

The elements of $G/H$ can be written $Hg$ where $g$ is an element of $G$. The operation on $G/H$ is defined as $Ha*Hb=Hab$ where $ab$ is defined by the operation of $G$. Since this operation is commutative on the elements of $G$, this implies that $Hab=Hba$, which means $Ha*Hb=Hb*Ha$, so $G/H$ is abelian.\\

\textbf{11.18} Let $G$ be cyclic and let $H$ be a subgroup of $G$. Show that $G/H$ is cyclic.

As in the 11.17, the elements of $G/H$ can be written $Hg$ where $g$ is an element of $G$. The operation on two such elements $Ha$ and $Hb$ is defined as $Ha*Hb=Hab$ where $ab=a*b$ defined by the operation on $G$. Let $q$ be a generator for $G$. Since $q$ generates all elements of $G$, $Hq$ is a generator for $G/H$, which means $G/H$ is cyclic.\\

\textbf{11.29)} Show that if $G/Z(G)$ is cyclic then $G$ is abelian.\\

Let $G$ be a group such that $G/Z(G)$ is cyclic. Then there exists an element of $G/Z(G)$, call it $Hq$, which generates the whole group. By the definition of the operation on quotient groups, this implies $q$ generates at least as many elements as the index of $Z(G)$ in $G$. (Note that $q$ does not necessarily generate all of $G$, since there may be multiple elements which give the same cosets).

I may need a hint on this one. It's really late, so I'm going to go to bed. Hopefully I'll get this before the rewrites are due.

\end{document}
