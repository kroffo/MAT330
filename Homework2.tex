\documentclass{scrartcl}
\usepackage{amsmath,amssymb}
\setkomafont{disposition}{\normalfont\bfseries}

\title{Abstract Algebra}
\subtitle{Homework 2: 2.2, 3.3, 3.10}
\author{Kenny Roffo}
\date{Due February 6}
\begin{document}

\maketitle

(2.10) Let $G$ be the set of all $2\times 2$ matrices $\begin{pmatrix} a & b \\ -b & a \end{pmatrix}$ where $a,b \in \mathbb{R}$ and $a^2 + b^2\ne 0$. Show that $G$ forms a group under matrix multiplication:\\\\
    (i) Matrix multiplication is a binary operator \checkmark\\
    (ii) Matrix multiplication is associative \checkmark\\
    (iii) $\begin{pmatrix} 1 & 0 \\ 0 & 1 \end{pmatrix}$ is the identity \checkmark\\
    (iv) Existence of inverses? I claim yes:

\em Proof: \em We want to show all $2\times 2$ matrices $\begin{pmatrix} a & b \\ -b & a \end{pmatrix}$ where $a,b \in \mathbb{R}$ and $a^2 + b^2\ne 0$ are invertible. Recall the Invertible Matrix Theorem which includes that an $n\times n$ matrix is invertible if and only if its degterminant is nonzero. Let $A=\begin{pmatrix} a & b \\ -b & a \end{pmatrix}$ be a matrix in the set. Then the determinant of $A$ is $a^2+b^2$. By definition on the set, $a^2+b^2\ne 0$, so the determinant of our arbitrary matrix has determinant 0, and is thus invertible. Therefore all matrices in the set have inverses.\\

Thus all of the group axioms are satisfied so this set under matrix multiplication forms a group.\\

(3.3) Find elements $A, B, C$ of $GL(2,\mathbb{R})$ such that $AB=BC$ but $A\ne C$:
    Consider the matrices $A=\begin{pmatrix} 1 & 1 \\ 2 & 1 \end{pmatrix}, B=\begin{pmatrix} 1 & 2 \\ 3 & 4 \end{pmatrix}, C=\begin{pmatrix} -3 & -4 \\ \frac{7}{2} & 5\end{pmatrix}$. Obviously $A\ne C$, but $AB=\begin{pmatrix} 4 & 6 \\ 5 & 8 \end{pmatrix}$.\\

(3.11) Let $(G,*)$ be a group such that $x^2=e$ for all $x \in G$. Show that $(G,*)$ is abelian.\\
\em Proof: \em We want to show $(G,*)$ is abelian. That is, we want to show $*$ is commutative on $G$. Let $a$ and $b$ be elements of $G$. Consider the equation $(a*b)*(a*b)=e$. Multiplying (*ing) on the left of both sides, we have $a*a*b*a*b=a*e$ which simplifies to $b*a*b=a$. Now multiplying by $b$ on the right, we have $b*a*b*b=a*b$, which simplifies to $b*a=a*b$. Thus $*$ is commutative on $G$.

\end{document}
